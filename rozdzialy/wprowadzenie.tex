\chapter{Wprowadzenie}

Wybór tematu mojej pracy jest ściśle związany z aktualnie panującymi trendami we współczesnym świecie programowania. Technologie asynchroniczne są w języku Python obecne w coraz większym stopniu i cieszą się rosnącą popularnością wśród programistów tego języka. Powodów takiego stanu rzeczy jest co najmniej kilka. 

Bogactwo frameworków webowych z wbudowaną obsługą asynchroniczności, coraz więcej współbieżnych modułów tworzonych przez społeczność do obsługi dotychczasowych sekwencyjnych funkcjonalności Pythona i wreszcie - coraz obszerniejsze wsparcie tego typu podejścia we wbudowanych bibliotekach języka oraz  w nim samym. Wymienione czynniki sprawiają, że uważam zajęcie się tą tematyką za pożyteczne.

Na początku pracy chciałbym zająć się szerzej tematem równoległości oraz współbieżności, gdyż pomimo że ogólnie pojęcia te bywają używane wymiennie, w mojej pracy będę je rozróżniał. Stąd też w pierwszych rozdziałach przedstawiam pokrótce ogólne definicje tych dwóch terminów i staram się rzetelnie nakreślić ich różnice.

W dalszej części pracy przedstawiam język Python. W mojej pracy skupiam się na jego konkretnej implementacji, którą jest CPython Znajomość krótkiej historii tej technologii będzie niezbędna, by zrozumieć pewnie mechanizmy, które na dość długoa, a nawet po dziś dzień utrudniają pracę z Pythonem jako językiem asynchronicznym. Dopiero jego najnowsze wersje wprowadziły do samej składni języka elementy ułatwiające zrozumienie działania mechanizmów asynchronicznych.

Kolejny rozdział poświęcam właśnie wspomnianym na końcu ostatniego akapitu mechanizmom. Jest ich co najmniej kilka i każdy ma swoją specyfikę pracy, wady oraz zalety. Mechanizmy te to mianowicie wbudowane moduły threading, multiprocessing oraz asyncio i powiązane z tym ostatnim słowa kluczowe async i await.

W rozdziale przedostatnim chciałbym zaprezentować sposoby użycia mechanizmów asynchronicznych w kontekście aplikacji webowej. Pierwszym wyborem ze względu na popularność tego frameworka względem jego dość krótkiego czasu istnienia jest FastAPI - uważany za jeden z najszybszych pythonowych frameworków webowych, do tego natywnie obsługujący asynchroniczność. Dla porównania zaprezentuję jak podobne zastosowanie byłoby rozwiązane w bardizej tradycyjnym frameworku, którym to w ninejszej pracy będzie Django.

Na koniec przerpowadzone eksperymenty opiszę i przestawię w postaci wyników czasowych cyklu zapytanie-odpowiedź.

\textbf{Słowa kluczowe:} współbieżność, równoległość, aplikacje internetowe