\chapter{Python}

\section{Krótka historia}
Pomimo tego, że Python uważany jest za jeden z nowocześniejszych współczesnych języków programowania to jego historia sięga aż końcówki lat '80 XX wieku. Jego pierwsza konkretna implementacja została rozpoczęta w grudniu 1989 roku przez Guido van Rossuma w Narodowym Instytucie Badań Matematyki i Informatyki w Holandii. Python miał być następcą języka ABC nad którym Guido van Rossum pracował od 1983 roku zanim zajął się Pythonem.\footnote{ "The Making of Python". Artima Developer. \url{https://www.artima.com/articles/the-making-of-python}, dostęp 18.06.2021}

Python po ABC odziedziczył ideę stania się językiem, który będzie skierowany do odbiorców niekoniecznie będących programistami lub osobami odpowiedzialnymi w jakikolwiek sposób za tworzenie oprogramowania. Dziś wydaje się, że ta idea została osiągnieta, gdyż Python jest jednym z najczęściej używanych języków w nauce poza informatyką. Przykładowo na szacunkowe 8.2 miliona użytkowników Pythona aż 69\% to twórcy rozwiązań uczenia maszynowego i sztucznej inteligencji oraz badacze danych\footnote{ang. data scientists}.\footnote{\url{https://slashdata-website-cms.s3.amazonaws.com/sample_reports/ZAamt00SbUZKwB9j.pdf}, dostęp 16.06.2021}

Wersja Pythona z numerem 1.0 ujrzała światło dzienne w styczniu 1994 roku, ale wersja 0.9 istniała już od stycznia roku 1991 i posiadała system klas z dziedziczeniem, obsługę błędów, funkcje, oraz podstawowe typy danych takie jak ,,list'', ,,dict'' czy ,,str'', a także system modułów. W wersji 1.0 dodane zostały znane z języka Lisp funkcyjne narzędzie lambda, map, filter i reduce.

Wersja 2.0 wprowadziła jeden z charakterystycznych mechanizmów Pythona jakim jest list comprehension. Nie jest to całkowicie indywidualny twór twórców Pythona, gdyż podobne mechanizmy funkcjonowały - nomen omen - w funkcyjnych językach programowania takich jak SETL czy Haskell. To właśnie w trakcie tworzenia Pythona z dwójką z przodu zostały dodane takie elementy jak ,,nested scope'' czy ,,context manager'', pośród wielu innych zmian.

W roku 2014, gdy istniała już wersja 3 języka, zostało już oficjalnie ogłoszone, że Python 2.7 (czyli ostatnia wersja Python z dwójką z przodu) będzie wspierana do roku 2020 i zachęcano użytkowników języka do jak najszybszego przechodzenia na wersję 3. Ostatnie wydanie wersji 2 - dokładnie wersja 2.7.18 - została opublikowana 20 kwietnia 2020 roku i jest uważana za oficjalną datę zakończenia cyklu życia Pythona 2.

W międzyczasie od 3 grudnia 2008 roku równolegle z wersją 2 istniała i była rozwijana wersja 3 Pytona.\footnote{\url{https://www.python.org/download/releases/3.0/}} Jest to wersja aktualnie używana i wspierana, która co warto podkreślić, zepsuła kompatybiilność wsteczną względem wersji 2 i sprawiła, że przejścia z poprzedniej wersji na nową potrafiły być w wypadku starszych aplikacji mocno czasochłonne. 

\section{Ogólny opis technologii}
Python jest językiem interpretowanym wysokopoziomowym ogólnego użytku. Przede wszystkim język ten kładzie nacisk na łatwość odczytu zapisanego w nim kodu. Łatwość tę osiągać ma poprzez użycie wcięć jako znaczących składniowo elementów języka.  Język ten jest dynamicznie typowany oraz korzysta z wbudwoanego ,,garbage collectora''. Wspiera co najmniej kilka paradygmatów programowania, w związku z czym pisząc w nim możemy używać na przykład paradygmatu proceduralnego, obiektowego czy też funkcyjnego.

Bardzo rozbudowana biblioteka standardowa umożliwia szeroki zakres pracy z językiem, a dodatkowo społeczność cały czas tworzy i nadzoruje tworzenie modułów gotowych do pobrania za pomocą narzędzia pip. Dostępne do pobrania stworzone przez społeczność technologie możemy znaleźć na stronie https://pypi.org/.\footnote{\url{https://pypi.org/}} Jak już wspominałem wcześniej popularność w dziedzinach związanych z analizą danych, uczeniem maszynowym czy sztuczną inteligencją Python zawdzięcza właśnie między innymi olbrzymiej społeczności tworzącej wysokiej jakości, otwarto-źródłowe narzędzia dostępne dla każdego.

\section{Global Interpreter Lock}
Ważnym dla mojego tematu pracy elementem Pythona jest jego Global Interpreter Lock. W skrócie jest to mutex (czy też semafor), który pozwala utrzymać kontrole nad interpreterem Pythona dokładnie jednemu wątkowi, co oznacza ni mniej, ni więcej, że tylko jeden wątek może być wykonywany w dowolnym pojedynczym momencie. Dopóki piszemy programy sekwencyjne nie ma to absolutnie żadnego wpływu na naszą pracę, jednak zaczyna sprawiac problemy wydajnościowe, gdy próbujemy nasz kod przestawić na działanie współbieżne.

Przede wszystkim jednak GIL, nie wziął się znikąd. Rozwiązywał on poważny problem zarządzania pamięcią. Każdy obiekt w Pythonie ma swój licznik odniesień, który wyznacza kiedy dany obiekt może być ,,sprzątnięty'' przez ,,garbage collector''. Jeśli dwa wątki zmieniałyby jednocześnie wartość tego licznika bardzo łatwo byłoby doprowadzić do sytuacji, w której obiekt nie zostałby nigdy usunięty lub - co gorsza - usunięty przedwcześnie. Łatwo sobie wyobrazić, że zaistnienie tego typu sytuacji może bardzo szybko doprowadzić do wycieków pamięci lub różnych niepożądanych błędów w naszym programie.

Z tych właśnie powodów zdecydowano się na implementację Global Interpreter Lock. Można było zastosować potencjalne, nieco bardziej skomplikowane rozwiązania jak ,,lock per object'', lecz byłyby one mniej wydajne lub mniej intuicyjne, zaś prostota Global Interpreter Locka wpłynęła bardzo pozytywnie na popularność Pythona, a jak twierdzą niektórzy,\footnote{\url{https://www.youtube.com/watch?v=KVKufdTphKs&ab_channel=AlphaVideoIreland}} jest on jednym z powodów dla których Python jest dziś tak bardzo popularny.